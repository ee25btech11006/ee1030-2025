\documentclass{beamer}
\usepackage[utf8]{inputenc}

\usetheme{Madrid}
\usecolortheme{default}
\usepackage{amsmath,amssymb,amsfonts,amsthm}
\usepackage{mathtools}
\usepackage{txfonts}
\usepackage{tkz-euclide}
\usepackage{listings}
\usepackage{adjustbox}
\usepackage{array}
\usepackage{gensymb}
\usepackage{tabularx}
\usepackage{gvv}
\usepackage{lmodern}
\usepackage{circuitikz}
\usepackage{tikz}
\lstset{literate={·}{{$\cdot$}}1 {λ}{{$\lambda$}}1 {→}{{$\to$}}1}
\usepackage{graphicx}

\setbeamertemplate{page number in head/foot}[totalframenumber]

\title{5.4.1}
\date{September 27,2025}
\author{ADUDOTLA SRIVIDYA-EE25BTECH11006}

\begin{document}

\frame{\titlepage}

\begin{frame}{Question}
Using elementary transformations, find the inverse of the following matrix. 
\begin{align*}
    \myvec{2&&3\\-4&&-6}
\end{align*}
\end{frame}

\begin{frame}{Theoretical Solution}
using row reductions:
\begin{align}
    \myvec{2&3\\-4&-6} 
    \xrightarrow[]{R_2 \rightarrow \frac{R_2}{-2}} 
    \myvec{2&3\\2&3}
\end{align}

\begin{align}
    R_2 \rightarrow R_2 - R_1 \implies \myvec{2&3 \\ 0&0}
\end{align}
since one row becomes zero, the matrix cannot be reduced to the identity matrix
\end{frame}

\begin{frame}{conclusion}
Hence, the given matrix is \textbf{not invertible}. 
\end{frame}


\begin{frame}[fragile]
\frametitle{Python,C,Python+C codes}
codes permalink
\end{frame}

\end{document}